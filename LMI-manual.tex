\documentclass[]{book}
\usepackage{lmodern}
\usepackage{amssymb,amsmath}
\usepackage{ifxetex,ifluatex}
\usepackage{fixltx2e} % provides \textsubscript
\ifnum 0\ifxetex 1\fi\ifluatex 1\fi=0 % if pdftex
  \usepackage[T1]{fontenc}
  \usepackage[utf8]{inputenc}
\else % if luatex or xelatex
  \ifxetex
    \usepackage{mathspec}
  \else
    \usepackage{fontspec}
  \fi
  \defaultfontfeatures{Ligatures=TeX,Scale=MatchLowercase}
\fi
% use upquote if available, for straight quotes in verbatim environments
\IfFileExists{upquote.sty}{\usepackage{upquote}}{}
% use microtype if available
\IfFileExists{microtype.sty}{%
\usepackage{microtype}
\UseMicrotypeSet[protrusion]{basicmath} % disable protrusion for tt fonts
}{}
\usepackage[margin=1in]{geometry}
\usepackage{hyperref}
\hypersetup{unicode=true,
            pdftitle={User Manual on Labor Market Information},
            pdfauthor={Institute for Community Inclusion},
            pdfborder={0 0 0},
            breaklinks=true}
\urlstyle{same}  % don't use monospace font for urls
\usepackage{natbib}
\bibliographystyle{plainnat}
\usepackage{longtable,booktabs}
\usepackage{graphicx,grffile}
\makeatletter
\def\maxwidth{\ifdim\Gin@nat@width>\linewidth\linewidth\else\Gin@nat@width\fi}
\def\maxheight{\ifdim\Gin@nat@height>\textheight\textheight\else\Gin@nat@height\fi}
\makeatother
% Scale images if necessary, so that they will not overflow the page
% margins by default, and it is still possible to overwrite the defaults
% using explicit options in \includegraphics[width, height, ...]{}
\setkeys{Gin}{width=\maxwidth,height=\maxheight,keepaspectratio}
\IfFileExists{parskip.sty}{%
\usepackage{parskip}
}{% else
\setlength{\parindent}{0pt}
\setlength{\parskip}{6pt plus 2pt minus 1pt}
}
\setlength{\emergencystretch}{3em}  % prevent overfull lines
\providecommand{\tightlist}{%
  \setlength{\itemsep}{0pt}\setlength{\parskip}{0pt}}
\setcounter{secnumdepth}{5}
% Redefines (sub)paragraphs to behave more like sections
\ifx\paragraph\undefined\else
\let\oldparagraph\paragraph
\renewcommand{\paragraph}[1]{\oldparagraph{#1}\mbox{}}
\fi
\ifx\subparagraph\undefined\else
\let\oldsubparagraph\subparagraph
\renewcommand{\subparagraph}[1]{\oldsubparagraph{#1}\mbox{}}
\fi

%%% Use protect on footnotes to avoid problems with footnotes in titles
\let\rmarkdownfootnote\footnote%
\def\footnote{\protect\rmarkdownfootnote}

%%% Change title format to be more compact
\usepackage{titling}

% Create subtitle command for use in maketitle
\newcommand{\subtitle}[1]{
  \posttitle{
    \begin{center}\large#1\end{center}
    }
}

\setlength{\droptitle}{-2em}
  \title{User Manual on Labor Market Information}
  \pretitle{\vspace{\droptitle}\centering\huge}
  \posttitle{\par}
  \author{Institute for Community Inclusion}
  \preauthor{\centering\large\emph}
  \postauthor{\par}
  \predate{\centering\large\emph}
  \postdate{\par}
  \date{2017-11-28}

\usepackage{booktabs}
\usepackage{longtable}

\usepackage{amsthm}
\newtheorem{theorem}{Theorem}[chapter]
\newtheorem{lemma}{Lemma}[chapter]
\theoremstyle{definition}
\newtheorem{definition}{Definition}[chapter]
\newtheorem{corollary}{Corollary}[chapter]
\newtheorem{proposition}{Proposition}[chapter]
\theoremstyle{definition}
\newtheorem{example}{Example}[chapter]
\theoremstyle{definition}
\newtheorem{exercise}{Exercise}[chapter]
\theoremstyle{remark}
\newtheorem*{remark}{Remark}
\newtheorem*{solution}{Solution}
\begin{document}
\maketitle

{
\setcounter{tocdepth}{1}
\tableofcontents
}
\chapter*{Introduction}\label{introduction}
\addcontentsline{toc}{chapter}{Introduction}

This user manual is an open-source living document that provides
description about how VR can use and potentially benefit from using
Labor Market Information (LMI). This manual provides information about
what LMI is, why it should be used, how it can be used and then provides
a catalogue of different sources of LMI. This manual also highlights how
some agencies are using LMI in innovative ways. Reason for keeping this
book open sourced, free to use, and free to edit is two folds. First,
LMI, its usage within VR are continously evolving. A standard format
document cannot keep-up with the changes that are happening on a daily
basis. Second, by open-sourcing this manual, we also want to invite VR
agencies, practitioners and researchers to contribue in making LMI more
usable and accessible for the VR. We understand that we havent been
exhaustive in listing all the sources and all the possible ways LMI can
be used. This manual should however guide its users in the direction
where they can find relevant LMI, make them comfortable using LMI and
finally enable them to contribute towards the common knowledge base.

\chapter{Purpose}\label{purpose}

The Demand Side RRTC partnered with the Alabama Department of
Rehabilitation Services beginning in 2012 to infuse labor market
information into a dashboard for counselors and to assist the ADRS to
expand the programmatic use of formal labor market data to better serve
jobseekers, understand business needs, and project long-term training
solutions. Early on in the effort, it became very apparent that there
are many sources of data and each have advantages and limitations for
use in VR. The RRTC created this catalogue as a guide for selecting data
elements and understanding the range of possible data sources. Since
2012, Congress passed the Workforce Innovation and Opportunity Act of
2014 that requires state VR agencies to use LMI. We expanded this
catalogue and made it available to all state VR agencies as a handy
reference for exploration. The Demand Side RRTC partnered with the RSA
funded Job Driven Vocational Rehabilitation Technical Assistance Center
to solicit review and recommendations to expand the catalogue. We
included more information about how to access LMI within different
states and provided some sample analyses that may be relevant for VR
agencies.

This catalogue is divided in two sections. In the first section we will
describe LMI using common definitions and discuss different types of LMI
and different ways of appreciating LMI. In the second section we
catalogue different sources of LMI available at the time fo this report.
The catalogue highlights critical LMI sources that have a high
likelihood of relevance to state VR agencies. More data sources are
likely to be available over time and will be added to the catalogue.

Note: We make no conclusion about whether or not any given data source
should be or should not be used by VR. We assert that VR agencies should
select sources that have use for the particular goal in mind. Yet, we
acknowledge that with an increase in readily available LMI data, there
is a high likelihood that VR personnel may become overwhelmed with
options. We hold to the belief that wise use of data is better than
simply increasing the number of data sources used.\\
Hopefully this manual will help VR to tease out relevant information
from a multitude information available to them.

\chapter{What is LMI?}\label{what-is-lmi}

According to the Bureau of Labor Statistics (2017), LMI is essential for
tracking and analyzing the economy of a country. National and local
governments need labor market information to reduce unemployment,
generate employment, or plan training programs to meet the needs of
industry. It is also used in determining future workforce training
needs, identifying the availability of labor, ascertaining prevailing
wage rates, exploring potential markets. LMI encompasses a wide array of
measures that can help describe an economy. Some of the primary examples
are unemployment rates, occupational statistics and employment
projections. Below is the list of common domains that tend to fall in
broad categories such as general economic indicators, labor force
descriptors, and occupational skills clusters.

\begin{itemize}
\tightlist
\item
  Employment and unemployment data or forecasts
\item
  Wage and hours worked data
\item
  Industry sector data including trends, such as job growth or decline
\item
  Occupation data including trends such as job growth or decline
\item
  Labor turnover and mobility
\item
  Average hours worked per week
\item
  General economic trends such as gross domestic product (GDP).
\end{itemize}

Arguably, there are three major types of sources of LMI.

\begin{itemize}
\tightlist
\item
  Traditional LMI
\item
  Real-Time LMI
\item
  Labor Market Intelligence
\end{itemize}

\section{Sources of LMI}\label{sources-of-lmi}

Different sources of LMI mostly gather data in different ways and oftne
provide exclusively different types of information. Therefore, it is
important to appreciate their difference and the value they bring to the
VR and decision making.

\subsection{Traditional LMI}\label{traditional-lmi}

Traditional LMI can be defined as a broad array of data on the labor
market that is systematically collected, analyzed and reported by the
government agencies. This data is made available free of cost by the
government. Traditional LMI sources covers wide range of geographical
area, depending upon which level of government collected the data. Key
data metrics include current employment levels, projected employment
growth, unemployment rates, average wages, minimum education
requirements, industry trends, and workforce demographics.

\subsection{Real-Time LMI}\label{real-time-lmi}

Real-time LMI is the data generated from job postings posted online.
Such information is derived from job-postings or resumes that are posted
online and can be obtained at a very granular level for different
localities. Real-time LMI is ``scraped'' concurrently from different
sources rather than compiled over long periods of time from surveys or
other data collection methods. Like traditional LMI, real-time LMI can
provide information on labor market trends such as wages, skills in
demand and emerging occupations or sectors.

\subsection{Labor Market Intelligence}\label{labor-market-intelligence}

Labor Market Intelligence is data gathered by VR employees from
employers, chambers of commerce, and industry associations about
opportunities and challenges of hiring for different positions, expected
changes in employment, and positions that require specialized skills.
Labor Market Intelligence can often be very local in nature -- i.e.,
providing labor market information for a particular set of employers in
a town. In comparison, Traditional LMI has to be aggregated at
relatively larger geographical unit. Labor Market Intelligence generated
by VR employees uses existing social networks by including different
stakeholders.

\chapter{Why Labor Market Information May Be Useful For State VR
Agencies}\label{why-labor-market-information-may-be-useful-for-state-vr-agencies}

State VR Agencies may find multiple uses for LMI at the administrative,
field services, and counselor levels. With the passage of WIOA, state VR
agencies and technical assistance providers are searching for ways to
integrate LMI into their work. State VR agencies such as the Alabama
Department of Rehabilitation Services are testing different methods.

LMI provides a big picture of labor market trends at the national, state
and local levels, and projects future outlook. By using LMI VR agencies
may be able to more effectively prepare demand-side strategic efforts
and respond to labor market needs.

VR agencies could use LMI to:

\begin{itemize}
\tightlist
\item
  Serve clients with disabilities, while considering the following
  points:

  \begin{itemize}
  \tightlist
  \item
    Demographic characteristics of the population
  \item
    Prevalent wages for certain occupation and industry
  \item
    In-demand skills and education
  \item
    Need for re-location and commuting
  \end{itemize}
\item
  Meet employer's needs, while keeping the following points into
  account:

  \begin{itemize}
  \tightlist
  \item
    Number of openings and mean salaries by industry and occupation
  \item
    Job openings during a particular period of the year
  \item
    Develop business partnership with key business players
  \end{itemize}
\item
  Improve strategic planning efforts
\end{itemize}

\begin{verbatim}
## 
## Attaching package: 'dplyr'
\end{verbatim}

\begin{verbatim}
## The following objects are masked from 'package:stats':
## 
##     filter, lag
\end{verbatim}

\begin{verbatim}
## The following objects are masked from 'package:base':
## 
##     intersect, setdiff, setequal, union
\end{verbatim}

\chapter{VR Roles and LMI Usage}\label{vr-roles-and-lmi-usage}

Different roles within a VR agency will use LMI in different ways. In
the following tables we are providing some examples of different roles
within an agency will use LMI based on their objectives and possible
benefits. We have classified five roles within VR to illustrate these
examples of using LMI-- * Counselor * Business Relations Representative
* Job Developers * Field Services Directors * Director

\section{Counselors using LMI}\label{counselors-using-lmi}

\begin{longtable}[t]{lll}
\caption{\label{tab:unnamed-chunk-2}VR Counselor: Possible examples of using LMI}\\
\toprule
Objective & Benefits & Example\\
\midrule
Assisting a jobseeker identify an employment goal & Jobseeker has information about the job they desire, the labor market, skill needs, job openings and future trends. & IPE includes services such as training or skills acquisition that match stated needs of employers in the industry of choice.\\
Understanding future trends in specific occupations or industries of interest to the jobseeker & LMI can give the best possible projections about whether or not an industry or occupation is likely to grow or decline in the local area of the jobseeker & Jobseeker may benefit from understanding what is likely to be available in an occupation if they chose to attend long-term training.\\
\bottomrule
\end{longtable}

\section{Business Relations Representative using
LMI}\label{business-relations-representative-using-lmi}

\begin{longtable}[t]{lll}
\caption{\label{tab:unnamed-chunk-3}VR Business Relations Representative: Possible examples of using LMI}\\
\toprule
Objective & Benefits & Example\\
\midrule
Develop business outreach strategies that address immediate needs of business community in the local area & Use of different types of LMI may give clues about what businesses are expanding, declining, laying off, or relocating. & Business relations personnel may identify a business that is rapidly expanding in a new area of the state and work with the VR offices to respond.\\
Maintain and improve connections with key businesses in the local economy & Using LMI may help a BR representative determine likely trends for industries or occupations and assist businesses with anticipating response for labor needs. & BR representative establishes partnerships with providers and other entities to provide continuous customer service to a business over time.\\
Identify business engagement strategies that are tailored to VR jobseekers with limited skills & Use LMI to identify occupations and industries that may have career advancement & Establish relationships with businesses that tend to hire entry- level workers with limited skills and then work with the employer to establish career advancement strategies.\\
\bottomrule
\end{longtable}

\section{Job Developers using LMI}\label{job-developers-using-lmi}

\begin{longtable}[t]{lll}
\caption{\label{tab:unnamed-chunk-4}Job Developers: Possible examples of using LMI}\\
\toprule
Objective & Benefits & Example\\
\midrule
Target job development to immediate needs of employers & Find open positions as they get posted through examination of real-time LMI & Identify employers that have significant hiring needs and establish relationships to identify talent prior to posting of positions.\\
Explore career ladders for entry-level jobs & Use occupation information for guidance about career progression options for certain occupations. & Create long-term relationships with employers by rapidly identifying and partnering with VR counselor and BR rep on establishing career pathways.\\
\bottomrule
\end{longtable}

\section{Field Services Directors using
LMI}\label{field-services-directors-using-lmi}

\begin{longtable}[t]{lll}
\caption{\label{tab:unnamed-chunk-5}Field Services Directors: Possible examples of using LMI}\\
\toprule
Objective & Benefits & Example\\
\midrule
Align VR workforce to accommodate the local/regional labor market & Use LMI to identify locations in the state in which there may be need for investment in business outreach and work with Business Relations Representatives to build capacity. & Align field services jobseeker services with business relations capacity by identifying areas for capacity building. Investigate needs for developing vendor capacity in areas of need.\\
Identify short-term and long-term training partnerships & Use LMI to identify occupations in high demand and skill requirements. & Identify long-term and short-term training that are matched with local labor market needs and jobseeker goals.\\
\bottomrule
\end{longtable}

\section{Directors using LMI}\label{directors-using-lmi}

\begin{longtable}[t]{lll}
\caption{\label{tab:unnamed-chunk-6}Director: Possible examples of using LMI}\\
\toprule
Objective & Benefits & Example\\
\midrule
Identify career pathways and workforce system partnership opportunities & Use LMI to get information about the critical labor needs of employers who are working with workforce systems to create career pathways. & Evaluate the opportunities for jobseekers to participate in career pathways established by the workforce system. Define strategies for partnership across agencies.\\
Evaluate how business relations personnel are identifying unique employment opportunities and identify employers who are active partners with the VR agency & Use Business intelligence generated by the BR unit to identify businesses that are partnering with VR. & Establish more in-depth and strategic partnerships with active businesses and request their assistance in advising on long-term development strategies.\\
\bottomrule
\end{longtable}

\chapter{LMI Data Source}\label{lmi-data-source}

In this section, we are introducing LMI data sources, the corresponding
dataset and their strengths and limitations for state vocational
rehabilitation agencies. This is not an exhaustive list and each state
may have additional sources. We have compiled them by public data
sources and private data sources. We have divided the data sources by
their ownership i.e., public data sources and private data sources. Most
of the data sources listed in this section are available for free.
However, some data sources are subscription based or require a fee. Most
of the description text, when available is taken from the websites that
describe the datasources. We intend to continual update this list with
the most recent links, new data sources and any other changes.

\chapter{Public Sources of LMI}\label{public-sources-of-lmi}

\section{The U.S. Department of
Labor}\label{the-u.s.-department-of-labor}

``The Bureau of Labor Statistics of the U.S. Department of Labor is the
principal Federal agency responsible for measuring labor market
activity, working conditions, and price changes in the economy.'' Its
mission is to collect, analyze, and disseminate essential economic
information to support public and private decision-making. The data
disseminated by the BLS is mainly gathered from surveys that are
administered monthly (e.g., Mass Layoff Statistics), quarterly (e.g.,
Quarterly Census of Employment and Wages), and annually (e.g., National
Longitudinal Surveys.) (U.S. Bureau of Labor Statistics 2017e).

\subsection{The Current Employment Statistics
(CES)}\label{the-current-employment-statistics-ces}

\url{http://www.bls.gov/ces/} (National Database)

\url{https://www.bls.gov/sae/data.htm} (State and Metro Area Database)

Current Employment Statistics (CES) survey, is based on a survey of
approximately 147,000 businesses and government agencies representing
approximately 634,000 worksites throughout the United States. The
primary statistics derived from the survey are monthly estimates of
employment, hours, and earnings for the Nation, States, and major
metropolitan areas. Preliminary National estimates for a given reference
month are typically released on the third Friday after the conclusion of
the reference period in conjunction with data derived from a separate
survey of households, the Current Population Survey (CPS). The reference
period for the CES survey is the pay period which includes the 12th of
the month. (U.S. Bureau of Labor Statistics 2017)

\textbf{Strengths:} The CES tracks hours worked and earnings (e.g.,
average hourly earnings, average weekly hours, and average weekly
overtime hours). CES data are classified according to the 2012 North
American Industry Classification System (NAICS), which streamlines the
industry classification process. CES data is updated every month thus VR
employees can track changes in the state labor market on a regular
basis.

\textbf{Limitations:} The CES collects data on non-agricultural
industries only. The data are aggregated at the industry level (e.g.,
average hourly earnings). The CES does not provide any county-level
data, and only metro areas covered in regional analysis. For a state VR
agency any analysis based on CES will be limited to the state level.

\textbf{Level of specificity:} The CES is published on a monthly basis
at the national level. This dataset gives an overview of the health of
the economy in the United States.

\subsection{The Quarterly Census of Employment and Wages
(QCEW)}\label{the-quarterly-census-of-employment-and-wages-qcew}

\url{http://www.bls.gov/cew/}

The QCEW is a complete count of employment and wages, classified by
industry and based on quarterly reports filed by employers for over 7
million establishments subject to unemployment insurance laws. This data
can be used to calculate the number of establishments, monthly
employment, and quarterly wages, by industry, by county, and for the
entire country.(U.S. Bureau of Labor Statistics 2017d)

\textbf{Strengths:} Original source of QCEW is the Quarterly
Contributions Report (QCRs) submitted to State Workforce Agencies (SWAs)
by employers subjected to unemployment insurance laws. This data is not
collected from individual institutions. Collected from SWAs this data is
an accurate reflection of UI data. The QCEW publishes a quarterly count
of employment and wages reported by employers covering 98 percent of U.S
jobs. State VR agencies can get granular data at a county level on
number of establishments , monthly employment, and quarterly wages.

\textbf{Limitations:} Wage data are available on a quarterly basis only.
The QCEW does not collect data from the self-employed, domestic workers,
and unpaid family workers.

\textbf{Level of specificity:} The information provided by this dataset
is available at the county, Metropolitan Statistical Area, state and
national levels by industry.

\subsection{The Local Area Unemployment Statistics
(LAUS)}\label{the-local-area-unemployment-statistics-laus}

\url{http://www.bls.gov/lau/}

LAUS is source of monthly and annual estimates for employment,
unemployment, and labor force data for census divisions, states,
counties, metropolitan areas, and cities. These data are available for
7,500 areas and can be used to assess localized labor markets and
perform comparative analysis across regions.(U.S. Bureau of Labor
Statistics 2017c)

\textbf{Strengths:} General strength of the LAUS is that it allows you
to identify unemployment trends at the country or city level (for some
states).

\textbf{Limitations:} The LAUS only looks at unemployment rates, and
cannot be used in conjunction with any other data source. For example,
one cannot find unemployment rates for any particular industry or
occupation. Therefore there is no way to connect it with any other
source of LMI.

\textbf{Level of specificity:} The LAUS produces data for census
regions, divisions, states, counties, metropolitan areas, cities with a
population of 25,000 or more and for all cities in the New England area.

\subsection{The Occupational Employment Statistics
(OES)}\label{the-occupational-employment-statistics-oes}

\url{http://www.bls.gov/oes/}

The OES program produces annual employment and wage estimates for over
800 occupations and 400 industries. BLS provides these estimates at
national, state and at metropolitan and nonmetropolitan geographical
levels. The OES also has occupational estimates for over 450 industry
classification. This data is collected by surveying approximately
200,000 establishments every six months. (U.S. Bureau of Labor
Statistics 2017c)

\textbf{Strengths:} The OES can be used to analyze occupational
employment, occupational wages, and occupational projections. The OES
can be used as tool by VR counselors during counseling by talking about
occupational trends in the state or local region. The OES can also be
used in conjunction with RSA 911 data by using Standard Occupation Codes
(SOC) as a key variable.

\textbf{Limitations:} Covers full-time and part-time non-farm industries
only and does not include self-employment, owners and partners in
unincorporated firms, household workers, or unpaid family workers.

\textbf{Level of specificity:} Estimates are available at the national
and state levels, and for selected metropolitan and nonmetropolitan
regions.

\subsection{The Mass Layoff Statistics
(MLS)}\label{the-mass-layoff-statistics-mls}

\url{http://www.bls.gov/mls/}

The MLS collects information on mass layoff actions. The information is
gathered using a combination of each state's unemployment insurance
databases and employer-provided data. Monthly mass layoff numbers are
collected from establishments from which a minimum of 50 claims for
unemployment insurance has been made during a five-week period. A
quarterly version of this dataset is also available.(U.S. Bureau of
Labor Statistics 2017d)

\textbf{Strengths:} MLS data are available on a monthly basis. In
addition to monthly reports, one can find quarterly and annual data. The
MLS can be used to look at the characteristics of dislocated workers by
their age, race, sex, ethnic group and place of residence. These
variables can help understand labor demand on monthly basis.
Additionally, this is the only source of data to find reasons for
separation between employee and employer, which could be useful to see
if any particular industry is performing poorly.

\textbf{Limitations:} MLS data are only a subset of all layoff activity.
Many layoffs fail to meet the qualifying criteria. For example, large
layoffs accompanied by low levels of unemployment insurance activity may
not be identified as an MLS event.

\textbf{Level of specificity:} Data generated from the MLS is available
at the state and industry level.

\subsection{Business Employment Dynamics
(BED)}\label{business-employment-dynamics-bed}

\url{http://www.bls.gov/bdm/home.htm}

BED data are generated from the Quarterly Census Employment and Wages
program. This data source includes quarterly measurements of gross job
gains and gross job losses. (U.S. Bureau of Labor Statistics 2017a)

\textbf{Strengths:} Using the BED, you can estimate gross job gains and
gross job losses in a particular establishment. Therefore, you can
estimate which establishments are opening, closing, expanding, or
contracting.

\textbf{Limitations:} This information is produced on a quarterly basis,
and is aggregate at national level. Data from the self-employed and
non-profit organizations are not included.

\textbf{Level of specificity:} These data are available in aggregated
form at the national and state level.

\subsection{The Job Opening and Labor Turnover Survey
(JOLTS)}\label{the-job-opening-and-labor-turnover-survey-jolts}

\url{http://www.bls.gov/jlt/}

This program under the BLS provides statistics on new openings, hires,
and separations. The JOLTS serves as a demand-side indicator of labor
shortages at the national level. (U.S. Bureau of Labor Statistics 2014)

\textbf{Strengths:} These data are released monthly. The JOLTS collects
data on total employment, job openings, hires, quits, layoffs and
discharges. It can be used for studying industry trends, education, and
job training purposes.

\textbf{Limitations:} It includes nonagricultural industries only.

\textbf{Level of specificity:} JOLTS data are available at the industry
level.

\subsection{The Current Population Survey
(CPS)}\label{the-current-population-survey-cps}

\url{http://www.bls.gov/cps/}

The CPS is a monthly household survey supported jointly by the U.S.
Census Bureau and the U.S. Bureau of Labor Statistics. It is the primary
source of labor market economic statistics, such as employment,
unemployment, and wages by industry, occupation, and demographic
characteristics. (U.S. Bureau of Labor Statistics 2017b)

\textbf{Strengths:} The CPS provides monthly employment and unemployment
data classified by a variety of demographic, social, and economic
characteristics. This includes information that may be pertinent to VR,
such as educational attainment and disability status.

\textbf{Limitations:} Disability status is self-reported, which may have
an impact on the quality of the data. Sampling strategies are complex
but were designed for national estimates and therefore reliable
estimates cannot be made at the county level. Many counties are not
included in the sample.

\textbf{Level of specificity:} CPS data are produced at the state level,
and for 12 of the largest metropolitan statistical areas.

\subsection{The National Compensation Survey
(NCS)}\label{the-national-compensation-survey-ncs}

\url{http://www.bls.gov/ncs/ect/}

The NCS is a survey conducted by the Office of Compensation Levels and
Trends, which provides the following statistics: - Quarterly changes in
employer costs: Employment Cost Index (ECI). The ECI shows changes in
wages and salaries, and benefit costs in addition to showing changes in
total compensation.

\begin{itemize}
\item
  Quarterly employer cost levels: Employer Costs for Employee
  Compensation (ECEC). The ECEC is a quarterly survey that shows total
  compensation costs using these variables: wages and salaries; total
  benefit costs; separate benefit costs for broad benefit categories
  such as paid leave, supplemental pay, insurance, retirement, and
  savings; and legally required benefits.
\item
  Incidence and provisions of employee benefits, such as the average
  number of paid holidays provided to employees each year. (U.S. Bureau
  of Labor Statistics 2013)
\end{itemize}

\textbf{Strengths:} The ECI presents data as a total for all workers,
and separately for private industry and for state and local government
workers. Unlike the CPS, which is administered to a member of a
household, this survey is administered to business establishments. Thus,
the NCS may help you determine what the acceptable wage is for a
particular position in a particular industry or occupational group. It
can provide insight on an employer that you may not be able to determine
by visiting their website. Occupations are classified using the Standard
Occupational Classification system.

\textbf{Limitations:} Not all occupations within a given establishment
are selected due to the method used in data collection by the NCS. The
NCS uses Probability Selection of Occupations as their method of choice.
It allows the NCS staff to select a particular occupation within the
establishment. Occupations with the highest number of individuals are
more likely to be selected.

\textbf{Level of specificity:} The ECI reports changes by industry,
occupational group, union and nonunion status, and census region and
division, for 15 large metropolitan areas and the nation.

\subsection{The Employment Projections
(EP)}\label{the-employment-projections-ep}

\url{http://www.bls.gov/emp/}

The EP program produces 10-year projections of industry and occupational
employment outlook. The latest projections are for the period of
2012--2022. This group also produces an annual Occupational Outlook
Handbook, which contains national-level analysis based on the employment
projections. (U.S. Bureau of Labor Statistics 2017a)

\textbf{Strengths:} The EP includes occupational profiles for the vast
majority of employment settings. It provides education, related work
experience, and on-the-job training requirements for a particular
occupation.

\textbf{Limitations:} Economic, employment, and labor force projections
are updated every other year. The EP does not produce short-term
projections. These projections are not to be confused with the short and
long-term projections compiled by your state's department of labor.

\textbf{Level of specificity:} Employment projections are provided at
the national level.

\subsection{O*NET Online}\label{onet-online}

\url{http://www.onetcenter.org/}

The Occupational Information Network (O*NET) program is a comprehensive
database of worker attributes and job characteristics and is compiled
and managed by the Employment and Training Administration. Its
interactive user interface contains standardized descriptions of
different occupations. Such data help with career exploration and
counseling, education, employment, training activities, and career
decision-making.

\textbf{Strengths:} The data collected and processed by O*NET is
available at no cost, and is continuously updated. The website also has
a suite of career exploration tools. These can be used by VR to assist
clients that want to choose a career or are considering a career change.

\textbf{Limitations:} O*NET data are heavily used by VR agencies and
vendor organizations. One disclaimer is that data or career assessments
included may have not undergone rigorous validity testing particularly
for persons with disabilities. Access to the related instruments
requires internet access and may require proficiency with information
technology.

\textbf{Level of Specificity:} These data are available at the regional,
state, and national levels.

\section{The United States Census Bureau and The U.S. Department of
Commerce}\label{the-united-states-census-bureau-and-the-u.s.-department-of-commerce}

The Census Bureau is the primary source of population demographics. The
Census conducts large-scale nationwide data collection with advanced
sampling methodologies that provide official national and
subnational-level measures.

\subsection{Longitudinal Employer-Household Dynamics
(LEHD)}\label{longitudinal-employer-household-dynamics-lehd}

\url{http://lehd.ces.census.gov/}

The LEHD program is part of the Center for Economic Studies at the U.S.
Census Bureau. Through the Local Employment Dynamics (LED) Partnership,
states have agreed to share unemployment insurance earnings data and
participate in the Quarterly Census of Employment and Wages. These data
are then combined with numerous administrative data, censuses, and
surveys in order to create statistics on employment, earnings, and job
flow in a specific geography and industry for different demographic
groups. These data are used to create anonymized data on workers'
employment, and the LEHD Origin-Destination Employment Statistics
(LODES). Industry-level data are available under the LEHD program, under
Industry Focus --
\url{http://lehd.ces.census.gov/applications/industry_focus.html}.

\textbf{Strengths:} The LEHD program produces data that can be used by
states and local authorities to make informed decisions about their
economies.

\textbf{Limitations:} Some of the data collected by the LEHD program is
considered to be confidential, and is only available to qualified
researchers with approved projects through restricted access use in
Census Research Data Centers.

\textbf{Level of specificity:} LEHD data are available at the state
level.

\subsection{The Economic Census}\label{the-economic-census}

\url{https://www.census.gov/programs-surveys/economic-census.html}

The Economic Census is the U.S. Government's official measure of
American business and economy. The Economic Census is the most
comprehensive source of information about American businesses from the
national to the local level. Published statistics include more than 1000
types of industries, 15,000 products, every state, over 3, 000 counties,
15,000 cities and towns, and Puerto Rico and other U.S. Island Areas.
The most recent Economic Census was conducted in 2012. Primarily the
economic census contains data on number establishments, revenue, annual
payroll (\$ value), and number of paid employees on payroll. This data
is available across all sectors. These data are generally used for
planning purposes by government agencies as well as by businesses. The
Economic Census is also used to evaluate market share and new business
opportunities.

\textbf{Strengths:} Economic Census data are used by businesses to
identify opportunities for growth. Having such an insight will allow VR
to target businesses in their state that are planning to expand. These
data can also be used to compare different communities.

\textbf{Limitations:} The Economic Census is conducted every five years,
in years ending in 2 and, and it does not census forms to most very
small firms.

\textbf{Level of specificity:} Data are typically available at six
levels of geographical specificity: National, States, Metro Areas,
Counties, Places, and at ZIP Codes level.

\subsection{Business Dynamics Statistics
(BDS)}\label{business-dynamics-statistics-bds}

\url{https://www.census.gov/ces/dataproducts/bds/}

The Business Dynamics Statistics (BDS) is a source of annual statistics
on the `business dynamics' (such as job creation and destruction,
establishment births and deaths, and firm startups and shutdowns) for
the economy and these data are aggregated by establishment and firm
characteristics. The Longitudinal Business Database (LBD)
\url{https://www.census.gov/ces/dataproducts/datasets/lbd.html} is the
source of BDS. The LBD is a confidential database available to qualified
researchers through secure Federal Statistical Research Data Centers
\url{https://www.census.gov/fsrdc}. Using BDS, one can track business
growth and decline in a given region.

\textbf{Strengths:} The BDS provides information on establishment
openings, job expansions, and number of startups within a region. This
information can help VR professions to identify industries that
expanding and creating more jobs in the state. Business relations
personnel at VR might find this data useful in identifying industry
sectors and establishments that are growing and are seeking skilled
employees.

\textbf{Limitations:} Self-employed, agricultural sectors, and most
government establishments are not included in this dataset.

\textbf{Level of specificity:} Data are available at the state,
metro/non-metro, and at Metropolitan Statistical Area level.

\chapter{Private Sources of LMI}\label{private-sources-of-lmi}

There are private for profit firms, non-profit organizations, and
research institutes that provide labor market data and analyses. The
information available through private sources is often available for
free and is mostly available on the internet. On some occasions value
added information, or information with higher fidelity is available only
through membership, subscription or for a fee. Several of the popular
sites that mostly function as a job board also provide statistics and
data on the labor market, along with career planning services. There is
a specific and growing sub-class of private LMI providers that focuses
mostly on providing real-time LMI. The field of private LMI in general
is continously evolving with consistent influx of new players. Real-Time
LMI can be considered as a special class of LMI that uses advanced
computational algorithms to collect and analyze data from job postings
across the internet. In comparison, conventional sources of private LMI
might gather data through surveys, providers of real-time LMI gathers
data from actual job postings from across the internet. Well known job
boards are also increasingly becoming sources for real-time LMI as they
increase their ability to parse job postings on their network as well as
on the internet in whole.

This catalogue is not endorsing any particular source but aims to
provide general information on various private sources of LMI. Users
should be aware of copyright or any other restriction with respect to
the information products or data available through private sources.

\section{Private sources of LMI}\label{private-sources-of-lmi-1}

As noted earlier there are multiple private sources of LMI. Some of the
companies collect data through surveys of employers. Other companies
that mainly function as job-boards analyzes the jobs posted on their
services and provide an aggregated outlook on trends in labor demand and
supply. These private sources also small briefs and reports that
synthesizes data that is either available through outher public LMI
sources or references the proprietary data.

\textbf{Strengths:} There are multiple sources of such private LMI to
chose from. VR counselors can chose the source that serves their needs
the best. Websites hosting private LMI also provides analysis and
synthesis of their data for easy consumption. Compared to public sources
of LMI, new set of private LMI data is made available at a higher
frequency.

\textbf{Limitations:} Private sources of LMI are often less exhaustive
than public sources of LMI in their reach and coverage. Private sources
of LMI or often not as extensively scrutinized as most of the public
sources of LMI are. In cases where data collection or analysis
methodology is proprietary, it is not easy to assess the methodology as
an user.

\textbf{Level of Specificity:} Most of the private sources of LMI are
frequently updated and are mostly available at state level. Specially,
the data based on job postings is available even for smaller regions
such as counties, cities and towns.

Following is a non-exhaustive list of private sources of LMI --

\begin{itemize}
\tightlist
\item
  \emph{Conference Board}
\end{itemize}

\url{https://www.conference-board.org/data/} ,
\url{https://www.conference-board.org/data/eti.cfm},
\url{https://www.conference-board.org/data/helpwantedonline.cfm}

Conference board is a global research association that provides
informational relevant to the businesses. The Conference Board website
provides publication and data related to the labor market. The
Conference Board also develops Employment Trend Index
\url{https://www.conference-board.org/data/eti.cfm}, which is based on
other publicly available sources of LMI. The Conference Board considers
its ETI as a short-term look of the labor market which can be used by
decision makers as a forecasting tool.

The conference board also produced Help Wanted Online (HWOL) Data Series
in partnership with Wanted Technologies. HWOL is monthly series that
looks at 16,000 internet job boards, corporate boards and other job
sites. HWOL provides seasonally adjusted labor supply and demand data
for the U.S., the nine Census regions, the 50 states, and locally for 52
largest metropolitan areas. HWOL data and related information is
available at
\url{https://www.conference-board.org/data/helpwantedonline.cfm}

\begin{itemize}
\tightlist
\item
  \emph{Manpower Inc.}
\end{itemize}

\url{http://www.manpower.com}

Manpower is a staffing agency that provides workforce management
services to employers. It conducts quarterly Employment Outlook Surveys
in the U.S. results of which are publicly available at
\url{http://manpowergroup.us/meos/}.

\begin{itemize}
\tightlist
\item
  \emph{Monster.com}
\end{itemize}

\url{http://www.monster.com}

Monster.com is an online job posting website. Monster.com provide salary
search to its consumers and also produce market intelligence reports on
labor statistics and trends, occupational reports and resources for job
seekers. Reports from Monster.com are available at -
\url{https://hiring.monster.com/hr/hr-best-practices.aspx}

\begin{itemize}
\tightlist
\item
  \emph{Indeed.com}
\end{itemize}

\url{http://www.indeed.com}

Indeed.com is a leading job search portal. Besides job postings, once
can find job category trends
(\url{https://www.indeed.com/jobtrends/category-trends}), salaries
(\url{https://www.indeed.com/salaries}) and company reviews
(\url{https://www.indeed.com/companies}).

\begin{itemize}
\tightlist
\item
  \emph{Careerbuilder}
\end{itemize}

\url{http://www.careerbuilder.com/}

Careerbuilder like Indeed.com is also a job search portal. Along with
providing access to job postings it also provides information salaries,
required skills and other resources. Most of the additional resources
are available at - \url{https://www.careerbuilder.com/insights/}

\section{Real-Time Labor Market
Information}\label{real-time-labor-market-information}

The real-time LMI as a sub-class of private LMI deserves discussion in
greater detail because of its scope and how it can compliment
traditional sources of LMI. Real-Time LMI is derived from online job
postings available across multiple job boards, both small and large.

Many a times online job postings contain wage range, industry
information and occupational information. Other information in job
postings such as position description, and skill requirements are also
valuable set of information. Organizations that provide real-time LMI
are continuously parsing online job postings and develop aggregates
based on region, industry type, occupation type and job requirements.
Most of the providers of LMI try their best to remove duplicates before
developing usable statistics.

\textbf{Strengths:} Real-time labor data provides users with scrubbed
online postings that are active at the moment and are representing
positions for which an employer is actively seeking workers. Many open
positions at one employer may signify change in that employers' hiring
needs. In general statistics generated through real-time LMI tools are
updates at least once in a day.

\textbf{Limitations:} Such data sources only collect job related
information that is available online. It is unclear if all labor market
sectors are actively posting online and whether an online posting is for
an open position. Some postings may be for positions in which there is
an internal hire already identified, or are always posted to ensure a
pool of applicants for when the position becomes available. Data
provided through such sources can be inconsistent because of the details
provided in various job postings.

\textbf{Level of Specificity:} Data are available at the city and often
at the zip code levels. Data are also available in real time. Details
such as skill requirements, educational requirements, experience
requirements, occupation codes, and sector information are also easily
available.

Following is a non-exhaustive list of sources of real-time LMI --

\begin{itemize}
\tightlist
\item
  \emph{Labor Insight by Burning Glass}
\end{itemize}

\url{http://burning-glass.com/labor-insight/}

Labor Insight by Burning Glass is an analytical tool that provide
aggregated labor market data in real-time. Burning glass aggregates and
analyzes job postings from multiple sources, remove the duplicates and
provides a dashboard to display relevant information.

\begin{itemize}
\tightlist
\item
  \emph{CEB TlentNeutron}
\end{itemize}

\url{https://www.cebglobal.com/talent-management/talent-neuron.html}

TalentNeutron, which is a part of Gartner was also known as Help Wanted
analytics. TalentNeutron is provider of real-time LMI analytics,
targeted mainly to HR services to help them make recruiting decisions.
It provides labor supply and demand information by region, industry,
occupational classifications and skills.

\begin{itemize}
\tightlist
\item
  \emph{Geographic Solutions}
\end{itemize}

\url{https://www.geographicsolutions.com/}

Geographic Solutions is another provider of real-time LMI. It has
developed America's Labor Market Analyzer (ALMA) as a tool that combines
multiple sources of LMI with real-time LMI.

\chapter{Existing Innovative
Strategies}\label{existing-innovative-strategies}

There are already some innovative strategies for LMI Integration used by
VR Agencies. There are examples of how VR agencies are using traditional
LMI, real-time LMI and labor market intelligence as a part of their
workflow. Following three examples use different sources of LMI to best
fit their needs.

\section{The Alabama Department of Rehabilitation Services (ADRS)
Dashboard}\label{the-alabama-department-of-rehabilitation-services-adrs-dashboard}

\url{http://www.rehab.alabama.gov/}

A dashboard is a tool that provides ``critical'' data accessible quickly
similar to GPS systems in cars that store massive amounts of data, but
pull relevant data to provide immediate directions. ADRS's dashboard is
integrating all three forms of LMI (Traditional, Real Time, and ADRS LMI
Intelligence Data).

\section{\texorpdfstring{Vermont Division of Vocational Rehabilitation's
(VDVR) ``Jobsville''
Model}{Vermont Division of Vocational Rehabilitation's (VDVR) Jobsville Model}}\label{vermont-division-of-vocational-rehabilitations-vdvr-jobsville-model}

\url{http://vocrehab.vermont.gov/}

As a part of VDVR's Jobsville model, VR counselors share client goals
with Business Specialists. Business Specialists provide the labor market
intelligence by sharing current employment opportunities in local
businesses.

\section{The Career Index Plus}\label{the-career-index-plus}

\url{https://www.thecareerindex.com/dsp_intro.cfm}

The Career Index Plus compares thousands of data points with the
customer's background, needs and circumstances in mind to produce a
unique and easy to understand summary of pros and cons along with
detailed support information for a given career choice. As a freely
available tool, The Career Index Plus can be used by a counselor as well
as person looking for employment. The Career Index Plus does not collect
or analyze any data by itself, it uses a variety of tools which it
integrates with real time local labor market information as well as
local job openings from thousands of online job boards and present the
data in an user friendly interface.


\end{document}
